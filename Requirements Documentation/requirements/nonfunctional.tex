%nonfunctional
% We have to decide on the utmost important req here and remove the others

\subsubsection{Performance}
For UPRM performance will not be a core non-functional requirement, however to make the system feasible and usable the following requirements must hold:

\begin{itemize}
	\item The response time of any query should be \textless{}  5 seconds.
	\item The generation of any report or statistic should be at most 10 seconds.
\end{itemize}

\subsubsection{Reliability}
Any system that works with live data such as UPRM should be reliable to ensure the integrity of the data stored in the database.\\
 
The following requirements must therefore hold, UPRM 

\begin{itemize}
	\item should never crash while updating, creating or removing research projects.
	\item should perform any functionality as precise as specified in the functional requirements at any moment during the use of the system.
\end{itemize}

\subsubsection{Availability}
Availability is a crucial non-functional requirement as the updating and creating of research projects is the most important requirement of UPRM.\\

Availability requirements are as follows,

\begin{itemize}
	\item UPRM have to be in a functional and working state for 95 - 99\% of the time per year. Downtime will then be between 3.45 - 18.25 days per year.
	\item UPRM may not exceed a continuous downtime for longer than 3 hours.
	\item MTTF of UPRM may not exceed 30 seconds.
\end{itemize}

\subsubsection{Security}
Security is one of the most significant non-functional requirements as the data that is processed and stored by UPRM is of a sensitive nature. 
A vast amount of research ideas and the progress status on current research will be stored by UPRM needless to say, if such data falls into the wrong hands it could jeopardise the research project or the idea of the research.\\  \\
\textbf{In terms of security}, UPRM should,
	\begin{itemize} 
		\item never reveal the current progress of research projects to unauthorized parties.
		\item never reveal future research ideas to unauthorized parties.
		\item prevent the loss of any data (personal and research related data). \\ \\
	\end{itemize}
\textbf{In terms of security requirements}, UPRM should,
	\begin{itemize} 
		\item make use of strong passwords.
		\item utilise two step verification/authentication.
		\item make sure that password resets is done by an administrator.
	\end{itemize}

\subsubsection{Maintainability}
	Although maintainability is usually an important non-functional requirement it is not one of UPRM's core non-functional requirements.\\ \\
Maintainability requirements are as follows,
\begin{itemize}
	\item Any defect must be corrected by a software developer with more than 1 year of experience within 5 days.
	\item All aspects of UPRM should be well documented to ensure faster feature addition or error fixing.
\end{itemize}


\subsubsection{Portability}
	UPRM will be used by a wide variety of users all with their own preferences. UPRM must therefore ensure portability of the system.\\ \\
	Portability requirements are as follows,
	\begin{itemize}
		\item The system should run on a desktop or laptop on the following,
			\begin{itemize}
				\item Linux(Ubuntu, OpenSUSE, Fedora, Debain, Elementary OS and Gentoo),
				\item Windows(7, 8, 8.1 and 10) and
				\item any IOS version.
			\end{itemize}
		\item The system should support Google Chrome, Firefox, Safari, IE(8, 9, 10, 11) and EDGE.
		\item  The system must be developed without a dedicated character-set, that is the system should run on computers independent of their character-set.
	\end{itemize}