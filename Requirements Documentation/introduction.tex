%introduction
\section{Introduction}

	\subsection{Purpose}
		This document serves as the software requirements specification (SRS) for UPRM (University of Pretoria Research Manager). 
		The requirements will include both the functional and non-functional requirements. 
		This document is written as a requirements guideline for the software enigineers and any other party involved in the creation of UPRM.

	\subsection{Scope}
		\begin{paragraph}{}
			The UPRM scope consists of a/an,
			\begin{itemize}
				\item research manager, known as RM.
				\item statistics query tool, known as RMSQ.
				\item event/venue creation tool known as RMVC.
				\item report generator tool, known as RMR.\\
			\end{itemize}
		\end{paragraph}

		\begin{paragraph}{}
			Description of each product within the scope of UPRM:
				\begin{itemize}
					\item RM -
						This will be the main focus for UPRM. This tool will assist the researcher with creating a new research "project". 
						The user/s will have to be able to create a new project, enter the research topic, create an approximate timeline for the completion of the paper which can then relate to percentage of completion. During the writing of the paper, the user should be able to change or edit the research topic. Once the user has marked the paper as complete, the user/s should be able to submit the paper to venue/s of their choice registered on the UPRM system.
					\item RMSQ -
						The statistics query tool has its own subset of tools combined into one system integrated with UPRM. The statistics query tool will have to be able to generate statistics at a global system level as well as at a user based level. With the statistics query tool we will have to factor in privacy issues and hence only certain functionality will be granted to certain users based on preferences set by the main UPRM user of project leader. On a \textbf{global level}, the user will have to be able to generate statistics on the following: 
							\begin{itemize}
								\item Percentage of paper acceptance for institution.
								\item Percentage of papers accepted per venue.
								\item Percentage of papers accepted for certain user/researcher.
								\item How many papers are currently being produced by the institution.
								\item How many papers are currently being produced by a certain user/researcher.
								\item Tracking the percentage of completion of a paper for a certain researcher.
							\end{itemize}
					\item RMVC -
						The venue creation tool will be the access point to external venue organisers such as conferences and research papers. This will assist the organisers of such venues to easily call on papers as well as assisting the researchers in submitting their research paper to multiple venues. With this functionality in the UPRM the user/s will have to be able to create, read, delete and update \textbf{CRUD} their own venues. The deadlines for submissions will have to be posted or indicated that there does not exist a deadline for submissions. The venues will also then, after papers have been submitted, be able to send feedback to the researchers on their paper as follow:
						\begin{itemize}
							\item Accepted (Published)
							\item Accepted (Not Published)
							\item Accepted with minor revisions
							\item Rejected
						\end{itemize}
					\item RMR -
						The report manager tool will generate exportable and distributable statistics in the format of the users choice that is supported by the UPRM RMR tool. This will be in the format of PDF or HTML or any other format the developer wishes to add. This tool will use the RMSQ tool to gather the statistics and will then export them to a new format as the RMSQ functionality only displays statistics which does not enable for distribution, unlike the RMR tool.
				\end{itemize}
		\end{paragraph}

	\subsection{Definitions, Acronyms and Abbreviations}
		\begin{itemize}
			\item{\textbf{UPRM}} - The system at hand, University of Pretoria Research Manager
			\item{\textbf{RM}} - Research Manager 
			\item{\textbf{RMSQ}} - Research Manager Statistics Query Tool
			\item{\textbf{RMVC}} - Research Manager Venue Creation
			\item{\textbf{RMR}} - Research Manager Reports
			\item{\textbf{MTTF}} - Mean-Time-To-Repair/Restart/Resume (Length of time that is needed to restart/repair or resume operation of the system after a failure.) 
			
		\end{itemize}
	\subsection{References}
	\nocite{*}
	\bibliographystyle{plain}
	\bibliography{references}

	\subsection{Overview}
	% Here we still need to state the chapters and appendices used
	% It might be better to finalize this section when most of the work is completed.
	% Example: http://www.cse.chalmers.se/~feldt/courses/reqeng/examples/srs_example_2010_group2.pdf
	The remainder of this document includes an overview of the UPRM system functionality and system interaction with other systems. The text will also introduce different types of stakeholders and their interaction with the UPRM system. Some system constraints and assumptions about UPRM will also be stated and explored upon.